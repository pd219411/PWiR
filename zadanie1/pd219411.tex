\documentclass[12pt,onecolumn,a4paper]{article}
\usepackage[utf8]{inputenc}

\usepackage[polish]{babel}
\usepackage[T1]{fontenc}
%\usepackage{ae,aecompl}

%\usepackage{color}
%\usepackage{xcolor}
%\usepackage{listings}

\usepackage{listings}

%\lstset{breaklines=true}

\begin{document}

\newpage
\onecolumn

\section{Opis}
Skrypt do weryfikacji modeli napisałem w pythonie, uruchamia się go bez parametrów.
Wszystkie modele znajdują się w jednym pliku. Model i formułę ltl wybieramy ustawiając odpowiednie wartości makrodefinicji.

\section{Wyniki}
Bezpieczeństwo jest zapewnione we wszystkich modelach niezależnie od implementacji.

Podobnie zawsze spełniona jest własność Concurrent Entering.

Żywotność nie jest zapewniona w przypadku wyboru słabego semafora, wtedy zwyczajnie głodzimy procesy na semaforze. Wybór implementacji Acquire jako bloku atomic nie popsuł tej własności, co było dla mnie zaskoczeniem.
Żywotność nie jest zapewniona również w przypadku wyboru niedeterministycznej kolejki. Pomimo dziedziczenia sekcji krytycznej, proces czekający w kolejce, może zostać zagłodzony przez procesy z innych grup, wskakujące do kolejki przed niego.

Jeśli w systemie są sami czytelnicy, to żaden czytelnik nigdy nie czeka i wynika to wprost z warunków w pierwszym bloku Acquire.

\end{document}